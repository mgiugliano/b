\chapter{Synapses and their Plasticity}
\label{syn} % Always give a unique label
\chaptermark{Synaptic Transmission}

\abstract{This chapter presents...}

\vspace{2cm} % Adds 1 centimeter of vertical space

\begin{svgraybox}
{\bf{Learning objectives}}
\begin{itemize}
	\item This is the first item.
	\item This is the second item.
	\item You can add as many items as you need.
	\item Here's another item.
  \end{itemize}
\end{svgraybox}

\clearpage


\section{Neurons communicate by means of synapses}
	\subsection{Describing the postsynaptic machinery of information transmission Chemical synaptic transmission: the “Ferrari” of information transport}
	\subsection{Electrical synaptic transmission: the “TESLA” of information transport}
	\subsection{Neuromodulation and tissue-wide orchestration}
	\section{Chemical synaptic transmission down to its minimal essence}
	\section{The release of neurotransmitter is captured by a mathematical model }
		\subsection{The binomial model of non-deterministic release }
		\subsection{The short-term dynamics of synaptic transmission}
	\subsection{Short-term depression: getting tired of “talking”}
	\subsection{Short-term facilitation: warming up and boosting confidence}
	\subsection{A non-deterministic version of homosynaptic plasticity}
	    \section{Long-term heterosynaptic plasticity changes transmission efficacy}
		\subsection{The firing-rate and the spike-timing dependent models of plasticity}
		\subsection{Connectivity motifs and synaptic plasticity}


		