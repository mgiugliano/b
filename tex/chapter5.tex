\chapter{Simulating or Modeling networks?}
\label{net} % Always give a unique label
\chaptermark{Networks}

\abstract{This chapter presents...}

\vspace{2cm} % Adds 1 centimeter of vertical space

\begin{svgraybox}
{\bf{Learning objectives}}
\begin{itemize}
	\item This is the first item.
	\item This is the second item.
	\item You can add as many items as you need.
	\item Here's another item.
  \end{itemize}
\end{svgraybox}

\clearpage


\section{As “LEGO bricks”, models of neurons and of synapses can be used to connect together neurons and simulate properties and behaviors emerging in a network.}
		\subsection{Balancing excitation and inhibition as hypothesized regime of operation in the cerebral cortical networks}
	\section{Connecting together a series of neural oscillators: the phase-response curve and synchronization}
	\section{Synaptic background activity and diffusion approximation of incoming synaptic inputs }
	\section{Declining individuality for team’s spirit: mean-field approximation}
	\section{Leaving behind action potentials }
	\section{Elementary primitives for computation: a network that amplifies, slow down, or “keep in mind” an external signal}
	\section{Critical phenomena in neuronal networks}
	\subsection{Modeling Tissues: pushing further upwards the level of description}
	
