\chapter{Dynamical Systems}
\label{dyn} % Always give a unique label
\chaptermark{Dynamics}

\abstract{This chapter presents...}

\vspace{2cm} % Adds 1 centimeter of vertical space

\begin{svgraybox}
{\bf{Learning objectives}}
\begin{itemize}
	\item This is the first item.
	\item This is the second item.
	\item You can add as many items as you need.
	\item Here's another item.
  \end{itemize}
\end{svgraybox}

\clearpage


\section{The analysis of simple dynamical systems leads to an intuitive understanding of excitability and electrical properties of neurons.}
	\subsection{What are the state variables and the parameters of a model?}
	\section{The action potential seen under a new light: geometry of state space}
	\section{The long term behavior of a system can be described in terms of the attractors}
	\section{The changes of activity behavior occur through bifurcations}
	\section{Single-dimensional dynamical systems}
	\section{Dynamical systems in two dimensions: the old Cartesian plane is useful after all}
	\section{Reinterpreting an action potential as a limit-cycle in a 2D system}
	\section{Grasping higher dimensional dynamical systems through lower dimensional projections}

