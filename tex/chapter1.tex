\chapter{Electrical Phenomenology of Neurons}
\label{phen} % Always give a unique label
\chaptermark{phenomenology}

\abstract{This chapter presents...}

\vspace{2cm} % Adds 1 centimeter of vertical space

\begin{svgraybox}
{\bf{Learning objectives}}
\begin{itemize}
	\item This is the first item.
	\item This is the second item.
	\item You can add as many items as you need.
	\item Here's another item.
  \end{itemize}
\end{svgraybox}

\clearpage


\section{Bioelectrical phenomena characterize the activity of neural systems}
\subsection{Electrophysiology measures single cell activities}
\subsection{Electroencephalography measures whole-brain electrical activity}


\section{Neurons produce bioelectricity, excitability and ionic flow}
\subsection{The tonic and the phasic electrical discharges in neurons}
\subsection{The action potential, the calcium spikes, and spikelets}
\subsection{Activity dependent changes of neuronal responses: adaptation}

\section{The action potential emerges from the interaction between positive and negative feedbacks}
